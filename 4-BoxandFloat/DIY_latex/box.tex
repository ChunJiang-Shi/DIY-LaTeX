\documentclass[11pt]{article}%文档类型

\usepackage[a4paper]{geometry}
\geometry{left=2.0cm,right=2.0cm,top=2.5cm,bottom=2.5cm}

\usepackage{ctex}
\usepackage{amsmath,amsfonts,graphicx,amssymb,bm,amsthm}
\usepackage{algorithm,algorithmicx}
\usepackage[noend]{algpseudocode}
\usepackage{fancyhdr}
%使用如下的宏包

\begin{document}
\section{盒子}
盒子是 排版的基础单元。\LaTeX 的输出页面本身就是一个大盒子,这个大盒子内部由很多的小盒子构成,每一行内容都是一个盒子,页面就是盒子套盒子的大盒子。

每个盒子的左侧均有一参考点 (Reference point)。盒子的基线(baseline)是通过参考点的一条水平线。当 \LaTeX 排版时,位于同一行盒子的参考点将被从左到右的排成一条直线,称为当前基线(current baseline),并使它与盒子的基线对齐。每个对象的参考点都被放置于当前基线上。

描述盒子主要是三个参数分别是高度,深度,宽度。高度是参考点到盒子顶部的距离,深度是参考点到盒子底部的距离,宽度则是盒子的宽度。\LaTeX 提供了一些实用的命令让我们生成一些有用的盒子。
\subsection{水平盒子}
水平盒子即水平对齐的盒子。
\subsubsection{生成盒子}
命令:
\begin{verbatim}
\mbox{文本} 和 \makebox[宽度][位置参数]{文本}
\fbox{文本} 和 \framebox[宽度][位置参数]{文本}
\end{verbatim}

左侧两条命令生成一个简单水平盒子,其宽度恰好是在{ }中给出的文本的宽度。fbox和framebox有外框,除此以外和前者相同。
右面两条命令生成由可省的长度参数来预先确定宽度的盒子文本的自然宽度要比所定义的宽度大的时候,就会出现突出。位置参数可以取居中c(默认值)、左对齐l、右对齐r 和分散对齐s。

 makebox命令,可在picture环境中生成一个居中的或者左右对齐的文本,也可以使部分文本重叠,如
\begin{verbatim}
\makebox[0pt][l]{/}S
\end{verbatim}
输出结果是\makebox[0pt][l]{/}S
\subsubsection{边框和调整}
\begin{verbatim}
\framebox[10em][r]{Test box}\\[1ex]
\setlength{\fboxrule}{1.6pt}
\setlength{\fboxsep}{1em}
\framebox[10em][r]{Test box}
\end{verbatim}
setlength命令可以调整边框的宽度fboxrule 和内边距fboxsep,注意长度需要带单位。
\subsubsection{保存和调用}
\begin{verbatim}
\newsavebox{\boxname}
\sbox{\boxname}{文本} or \savebox[宽度][位置]{文本}
\usebox{\boxname}
\end{verbatim}
以上三条命令可以初始化,保存和调用已有的盒子。

\subsubsection{盒子的竖直移位}
\begin{verbatim}
\raisebox{上移量}[高度][深度]{文本}
\end{verbatim}

\subsection{垂直盒子}
垂直盒子即垂直对齐的盒子, \LaTeX 提供了两种方式:
\begin{verbatim}
\parbox[整体位置参数][高度][内部位置参数]{宽度}{文本}
\begin{minipage}[整体位置参数][高度][内部位置参数]{宽度}
. . .
\end{minipage}
\end{verbatim}
和水平盒子不同,内部位置参数接受的选项是顶部t、底部b、居中c 和分散对齐s。

如果在minipage 里使用footnote 命令,生成的脚注会出现在盒子底部,编号是独立的,
并且使用小写字母编号。这也是minipage 环境之被称为“迷你页”(Mini-page)的原因。而在
parbox 里无法正常使用footnote 命令,只能在盒子里使用footnotemark,在盒子外使用
footnotetext。
\subsection{标尺盒子}
rule 命令用来画一个实心的矩形盒子,也可适当调整以用来画线。
\begin{verbatim}
\rule[位置浮动]{宽度}{高度}
A \rule[-.4pt]{3em}{.4pt} line.
\end{verbatim}
也可以生成一个宽度为零的标尺盒子,这样得到一个不可见的具有给定高度的盒子,如此的构造称为一个支撑,如此可以强迫所在的水平盒子具有不同于其所包含内容需要的高度和深度.
\section{浮动体}

\end{document}
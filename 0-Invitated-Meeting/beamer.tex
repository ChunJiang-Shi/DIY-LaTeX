\documentclass{beamer}

\usepackage[scheme=plain]{ctex}
\usepackage{enumerate}
\usepackage{enumitem}   % [inline] option for inline enumerate, following https://tex.stackexchange.com/a/146311
\usepackage{amsmath,amssymb}
\usepackage{mathtools}
\usepackage{xcolor}
\usepackage[backend=biber]{biblatex}
\usepackage{booktabs}
% \usepackage{tikz}ctexctexctexcctexctexctexctexctexctexctexctextex
% \usepackage[colorlinks]{hyperref} % Already loaded by previous package(s)

\usefonttheme{serif} % using non standard fonts for beamer
% Following https://tex.stackexchange.com/a/34267
% \usefonttheme[onlymath]{serif}

% \setbeamerfont{frametitle}{series=\bfseries}
\setbeamerfont{title}{series=\bfseries}
% '\mdseries' Follows from https://tex.stackexchange.com/a/320244
\setbeamerfont{subtitle}{series=\mdseries}

\usetheme{Luebeck}

\addbibresource{ref.bib}

\hypersetup{
    colorlinks,
    urlcolor=blue,
    linkcolor=cyan,
    citecolor=red,
}

\setitemize{label=\usebeamerfont*{itemize item}%
  \usebeamercolor[fg]{itemize item}
  \usebeamertemplate{itemize item}
}

\newcommand{\opt}[1]{{\color{gray} #1}}



\begin{document}
\title[UCAS \LaTeX er 邀请会]{%
    \begin{tabular}{r@{~}c@{~}l}
    \textbf{U\,$\heartsuit$\,CAS} &  & \textbf{\LaTeX er} \\[.5em]
    主题讨论 & $\bullet$ & 邀请会
    \end{tabular}%
}
\subtitle{
\begin{flushright}
暨 DIY \LaTeX 系列活动启动会 $\qquad$
\end{flushright}
}
\author[DIY \LaTeX]{\textbf{DIY \LaTeX}\\ \small \url{https://github.com/Memcys/DIY-LaTeX.git}}
\date{2019 年 9 月 8 日}

\maketitle

\frame{
    \tableofcontents
}
\section[简介]{\LaTeX 简介}
\subsection{\LaTeX 简介}
\begin{frame}{\LaTeX 简介}
\frametitle{\LaTeX 简介}
The \LaTeX{} Project\cite{latex-project} 自称:
\begin{quotation}
LaTeX is a \textbf{high-quality} typesetting system.

LaTeX is \textbf{free software} under the terms of the \href{https://www.latex-project.org/lppl/}{LaTeX Project Public License (LPPL)}.% LaTeX is distributed through CTAN servers or comes as part of many easily installable and usable TeX distributions provided by the TeX User Group (TUG) or third parties.
\end{quotation}

\LaTeX{} Wikibook\cite{latex-wikibook} 上称:
\begin{quote}
It is the \textbf{de-facto standard} for \emph{academic journals and books}, and provides some of the best typography free software has to offer. 
\end{quote}
\end{frame}

\subsection{\LaTeX 优缺点}
\begin{frame}{\LaTeX 优点}
(优缺点均参考 \cite{lshort-cn}):
\begin{itemize}
\item 专业的排版输出能力
\item 方便而强大的数学公式排版能力,无出其右者
\item 有专用模板时,用户只需专注于一些组织文档结构的基础命令,无需(或很少)操心文档版面设计
\item 很容易生成复杂的专业排版元素,如脚注、交叉引用、参考文献、目录等
\item 强大的可扩展性,拥有众多 \LaTeX 宏包
\item 能够促使用户写出结构良好的文档
\item 跨平台、免费、开源
\end{itemize}
\end{frame}

\begin{frame}{\LaTeX 缺点}
\begin{itemize}
\item 入门门槛高
\item 不易排查错误
\item 不易定制样式
\item 需编译后才能得到输出文档
\end{itemize}
\end{frame}

\section[竞答]{“一行代码”竞答}
\begin{frame}[allowframebreaks]{“一行代码”竞答}

所有公式参考自 \cites{Zorich,Kostrikin}

% Single-line codes that appear in the beamer.tex
{\small
请忽略每题的序号(序号仅作为标识)和空格符号 ``\verb*| |''(实现真正的空格即可)。仅考虑 \verb|\begin{document}| 与 \verb|\end{document}| 之间甚至是数学模式中的内容。除特殊声明外无需指出所用宏包。
}

\begin{equation}
	C_n^m = \frac{n!}{m!(n-m)!}
\end{equation}
\begin{equation}
	\forall x, x \in S \Rightarrow x \in T
\end{equation}
\begin{equation}
    % Following https://texblog.org/2007/08/27/number-sets-prime-natural-integer-rational-real-and-complex-in-latex/
	\mathbb{R}^2 = \mathbb{R} \times \mathbb{R}
\end{equation}
\begin{equation}
	\int_a^b f(x) \mathrm{d} x \coloneqq \lim_{\lambda(P) \to 0} \sigma(f, P, \xi)
\end{equation}
\begin{equation}
	\Delta_n = \prod_{1 \leqslant i < j \leqslant n} (x_j - x_i)
\end{equation}
\begin{equation}
	\boxed{\lim_{x \to \infty} (1 + \frac{1}{x})^x = e.}
\end{equation}
\begin{equation}
    \begin{aligned}
        x_{1, 2} &= \frac{-b \pm \Delta}{2a} \\
        \Delta &= \sqrt{b^2 - 4ac}
    \end{aligned}
\end{equation}

Hello,\verb*| |\LaTeX\verb*| |user!

CTEX $\neq$ \LaTeX

有解的方程组叫作\textbf{相容的}。 \hspace{1cm} (注:请自行添加所需宏包)\\ % Need additional package (ctex or xeCJK) for Chinese characters!
\end{frame}

\section[讨论]{讨论}
\subsection{后续活动}
\begin{frame}[allowframebreaks]{主题元素}
\begin{Form}
% Following https://tex.stackexchange.com/a/310574
    \begin{itemize}[label = {\CheckBox[width=.12in,height=.12in,bordercolor=blue]{}}]
    \item 排错寻助
        \begin{itemize}[label = {\CheckBox[width=.1in,height=.1in,bordercolor=cyan]{}}]
        \item \opt{*.log 编译日志}
        \item MWE (Minimal Working Example)
        \item 宏包文档阅读
        \item \TeX 论坛社区群组
        \end{itemize}
    \item 公式定理
        \begin{itemize}[label = {\CheckBox[width=.1in,height=.1in,bordercolor=cyan]{}}]
        \item 数学模式、AMS 宏集
        \item 行内、行间及多行公式
        \item \opt{公式间距、字体控制、定理环境}
        \end{itemize}
    \item 中文排版(xeCJK 宏包,ctex 宏包和文档类)
    \item 文献索引
        \begin{itemize}[label = {\CheckBox[width=.1in,height=.1in,bordercolor=cyan]{}}]
        \item biblatex / natbib
        \item bib 条目的获取、自动生成
        \item \opt{索引}
        \end{itemize}
    \item 文档元素
        \begin{itemize}[label = {\CheckBox[width=.1in,height=.1in,bordercolor=cyan]{}}]
        \item 表格和图片
        \item 章节、目录、标题、边脚注
        \item 列表、引用、摘要环境
        \item 交叉引用
        \item \opt{盒子、浮动体}
        \end{itemize}
    \item \opt{Beamer}
    \item \opt{Ti\emph{k}Z 绘图}
    \end{itemize}
\end{Form}
\end{frame}

\subsection{时间地点}
\begin{frame}{时间地点}
\begin{tabular}{rl}
时间 & 每周日上午 10:00\\
地点 & 玉泉图书馆报告厅\\
例外 & 中秋和国庆假期间活动暂停。 \\
& 即 9 月 15 日和 10 月 6 日暂停活动。 \\
\end{tabular}
\end{frame}

\subsection{活动机制}
\begin{frame}{轮流主讲制}
\begin{itemize}
\item 后续系列活动每次活动围绕一个主题展开
\item 每个主题由一至两人主讲
\end{itemize}
\end{frame}

\begin{frame}{个人积分制}
\begin{itemize}
\item 个人在活动中的提问、发言等均增加积分
\item 一次活动中同一人集中于同一话题的提问或发言计一分
\item 若提问引起热烈讨论,或被公认具有一定价值;或发言正确,或被公认具有启发性,则再增加一分
\item 最后一次活动现场公布积分最高的前几名
\item 期间个人可向活动负责人询问自己的积分情况
\end{itemize}
\end{frame}

\begin{frame}{奖励机制}
\begin{itemize}
\item 活动最后一期(或倒数第二期)公布获奖者
\item 奖励对象为特定人数的主讲人和个人积分高的其他参与者
\item 可获奖主讲人人数少于实际主讲人数时,在公布获奖前采取投票。票数高者可获奖
\item 奖品内容待定
\end{itemize}
\end{frame}

% \section{尾声}
\subsection{\TeX 发行版安装及使用}
\begin{frame}{\TeX 发行版}
\begin{table}
\caption{\TeX 发行版可列举如下(参考 \href{https://www.latex-project.org/get/\#tex-distributions}{The \LaTeX{} Project}):}
\centering
\begin{tabular}{cccc}
\toprule
Linux & Mac OS & Windows & Online \\ \midrule
\TeX Live & Mac\TeX & \TeX Live & Overleaf \\
& & MiK\TeX & Papeeria \\
& & pro\TeX t & Datazar \\
& & & \dots \\ \bottomrule
\end{tabular}
\end{table}
\end{frame}

% \subsection[安装]{安装 \TeX 发行版}
\begin{frame}{安装 \TeX 发行版}
我们建议本地安装 CTAN\cite{CTAN} 的 \TeX 发行版。安装前请阅读 \href{https://www.tug.org/texlive/quickinstall.html}{\TeX{} Live -- Quick install} 或 \href{https://www.tug.org/texlive/doc/texlive-en/texlive-en.html\#x1-180003.1.3}{The \TeX{} Live Guide}. 此外,可参考 \cite{install-latex} 中的文档。建议 Windows 用户使用命令行安装。

\begin{table}
\caption{\TeX{} Live 两种安装方式。$\ast$nix 系统源安装方式未列出}
\centering
\begin{tabular}{c*{2}{p{.35\textwidth}}}
\toprule
& 在线安装 & 离线安装 \\ \midrule
下载内容 & {安装器 17.8MB/4MB;按需下载宏包} & \href{http://mirror.ctan.org/systems/texlive/Images/}{完整程序包 iso} 3.3GB \\
耗时 & \multicolumn{2}{c}{安装较多宏包时,离线安装耗时显著短于在线安装} \\ \bottomrule
\end{tabular}
\end{table}
\end{frame}

% \subsection{首次使用 \LaTeX}
\begin{frame}{首次使用 \LaTeX}
可参见 \cite{lshort-cn} 中 \S{}1.2 或 \cite{install-latex} 中相应内容。

使用 \LaTeX 流程可概括如下:
\begin{itemize}
\item 撰写(需\textbf{编辑器}) .tex 文件并保存(如 filename.tex)。
\item 编译(需\textbf{编译命令}) .tex 文件\emph{足够次数},得到输出文件(例如 filename.pdf)。
\end{itemize}
接下来便可以打开浏览输出文件了。
\end{frame}

\subsection{结语}
\begin{frame}{结语}
\cite{Kostrikin} 末尾引用了 \emph{John} 16:12\cite{John-16}:
\begin{quote}
I have much more to say to you, more than you can now bear.
\end{quote}

当然,我也只不过是话中的 ``you''.
\end{frame}

\subsection{参考及推荐资料}
\begin{frame}[t,allowframebreaks]
\printbibliography[heading=none]
\end{frame}
\end{document}

\documentclass[final,aspectratio=169]{ctexbeamer}

% Custom package providing several listing environments
\usepackage{mylsts}
% Miscellaneous in preamble specific to Chinese-beamer.tex
\usepackage{mymisc-beamer} 
% \usepackage{syntonly}
% \syntaxonly

% Title info
\title{中文 \LaTeX 简介}
\subtitle{\textbf{DIY \LaTeX} $4$th Report}
\author[Memcys]{Memcys \\ \url{https://github.com/Memcys/DIY-LaTeX}}
\institute[UCAS]{University of Chinese Academy of Sciences}
\date{(\zhganzhinian{2019}) \zhdate{2019/10/27} \zhweekday{2019/10/27}}

\begin{document}
\frame{\titlepage}

\setcounter{section}{-1}

\section[回顾]{回顾:数学和提问相关}
\subsection{回顾}
\begin{frame}{前三次报告回顾}
前三次报告中,FZX、LH、GR、LFZ (姓名首字母缩写) 就
\begin{enumerate}
\item \LaTeX 概述
\item 数学环境
\item AMS 宏集
\item 字体
\item 提问
\end{enumerate}
等主题作了详细的介绍。请回忆相关内容。
\end{frame}

\subsection{补充}
\begin{frame}[fragile]{补充}
\begin{itemize}
  \item 带框数学环境 \verb|\boxed{}, \framebox{}|
  \item gathered, aligned, alignedat 嵌套数学环境
  \item cases 环境
  \item QQ/ 微信 \TeX 群组
  \item 中文问答网站 \url{https://wenda.latexstudio.net/}
  \item 命令行工具 texdoc
\end{itemize}
\end{frame}

\begin{frame}[fragile]{带框数学环境}
带框数学环境 \verb|\boxed{}, \framebox{}|. 以 \verb|\boxed{}| 为例:
\begin{vertlst}
% \usepackage{amsmath}
% \newcommand{\iu}{\mathrm{i}\mkern1mu}
% \newcommand{\e}{\mathrm{e}}

\boxed{\e^{\iu \theta} = \cos{\theta} + \iu \sin{\theta}}
\end{vertlst}
\end{frame}

\begin{frame}[fragile]{嵌套数学环境}
gathered, aligned, alignedat 嵌套数学环境。以 gathered 为例:
\begin{vertlst}
% \usepackage{amsmath}

\begin{equation}
\begin{gathered}[]
  [f, f] = 0, \\ [f, g] + [g, f] = 0, \\
  [f, [g, h]] + [g, [h, f]] + [h, [f, g]] = 0.
\end{gathered}
\end{equation} 
\end{vertlst}
\end{frame}

\begin{frame}[fragile]{cases 环境}
cases 环境。效果如下:
\begin{sidelst}
\begin{equation}
  \delta_{ij} =
  \begin{cases}
    0, \, & i \ne j \\
    1, & 1 = j
  \end{cases}
\end{equation}
\end{sidelst}
\end{frame}

\begin{frame}[fragile]
\frametitle{问答相关}
\begin{itemize}
\item QQ 或微信有不少 \TeX 用户群组。其中有些群组有活跃开发 \TeX 项目的人员,回答质量高,且群组里有较好的问答氛围。
\item \url{https://wenda.latexstudio.net/} 是一个新成立(至少基于一个五千人 QQ 群)但活跃的问答社区。一些优秀的 \TeX 开发人员也常在此回答问题。
\item 查看宏包的另一种方式是使用命令行工具 \textbf{texdoc}:
\begin{bashlst}
texdoc packagename
\end{bashlst}
当然,它实际上是已经介绍的 TeXDoc GUI 的底层实现。
\end{itemize}
\end{frame}

\section*{目录}
\begin{frame}{目录}
\begin{columns}
  \begin{column}{.6\linewidth}
  \tableofcontents[hideallsubsections]
  \end{column}
  \begin{column}{.4\linewidth}
  \pgfornament[width=.7\linewidth,color=blue!40!green!60,opacity=.2]{74}
  \pgfornament[width=.7\linewidth,color=blue!10,opacity=.25]{133}
  \end{column}
\end{columns}
\end{frame}

\section{ctex 宏包 + xelatex 编译}
\begin{frame}[fragile]{中文输入困境}
\texinputlst[label=lst:hello]{Demos/demo-hello.tex}
一个入门级别的 helloworld 程序。我用了四个月才解决中文的问题。
\end{frame}


\section{xeCJK 宏包}
\subsection[示例]{调用 xeCJK 宏包示例}
\begin{frame}[fragile]{xeCJK 宏包}
需要输出中文时,我们推荐调用 xeCJK 宏包,并相应使用 \hologo{XeLaTeX} 引擎编译。后文有更多的解决方案。
\texinputlst[label=lst:demo-xecjk]{Demos/demo-xeCJK.tex}
\end{frame}

\subsection[宏包选项]{xcCJK 宏包选项选讲}
\begin{frame}[fragile]{xcCJK 宏包选项选讲}
见 xeCJK 宏包 \cite{xecjk} \S 3.1:
\begin{itemize}
  \item xeCJKactive = (\textit{true}|\textbf{false}) \\
  xeCJK 默认忽略 CJK 文字间的空格,使用这一选项来保留这些空格。
  \item CJKmath = (\textit{true}|\textbf{false}) \\
  是否支持在数学环境中直接输入 CJK 字符。 \\
  url 宏包将一个 URL 放在一个特殊的数学环境中排版。启用该选项,\verb|\path| 等命令的路径中的汉字才能显示。
  \item AutoFallBack = (\textit{true}|\textbf{false}) \\
  当文档中有主字体不包含的字时,自动使用预设的后备字体来输出这些生僻字。
\end{itemize}
\end{frame}


\section{ctex 宏集}
\subsection{\CTeX 宏集的中文支持}
\begin{frame}
\frametitle{宏包 + 编译方式}
\begin{table}
\caption{\CTeX 宏集的中文支持方式}
\label{tb:ctex}
\centering
\begin{tabular}{*{5}{c}}
\toprule
编译方式 & (pdf)\LaTeX & \hologo{XeLaTeX} & \hologo{LuaLaTeX} & up\LaTeX \\ \midrule
支持宏包 & CJK & xeCJK & LuaTeX-ja & 原生 \\ \bottomrule
\end{tabular}
\end{table}
\CTeX 宏集会根据用户使用的编译
方式,在底层选择不同的中文支持方式。
\end{frame}

\subsection{ctex 文档类和宏包}
\begin{frame}
\frametitle{ctex 文档类和宏包}
\begin{table}
\caption{ctex 文档类和宏包}
\centering
\begin{tabular}{*{5}{c}}
\toprule
ctex 文档类 & ctexart & ctexrep & ctexbook & ctexbeamer \\ \midrule
对应标准文档类 & article & report & book & beamer \\ \bottomrule
\end{tabular}
\end{table}
ctex 宏集中现提供宏包 ctex.sty, ctexsize.sty, ctexheading.sty. ctex 文档类调用了 ctex 宏包。
\end{frame}

\subsection{\textbackslash ctexset\{\}}
\begin{frame}
\frametitle{\textbackslash ctexset\{\} 选讲}
\begin{itemize}
  \item scheme = (\textbf{chinese}|\emph{plain})
  \begin{description}
    \item[chinese] 调整默认字号;汉化标题名字(如“表”等);在 heading = true 时,将章节标题修改为中文样式。
    \item[plain] 仅提供中文支持功能,而不对文章版式进行任何修改。
  \end{description}
\item linespread = (数值)\\
设置行距倍数。初始值依赖于 scheme.
\item today = (\textbf{small}|\emph{big}|\emph{old})
\begin{description}
  \item[small] 效果为“2019 年 10 月 27 日”
  \item[big] 效果为“二〇一九年十月二十七日”
  \item[old] 效果为原本的(英文) 日期格式 
\end{description}
\end{itemize}
\end{frame}

\subsection{实用命令选讲}\label{sec:practical-options}
\begin{frame}[fragile]
\frametitle{实用命令选讲}
\begin{itemize}
\item \verb|\zihao{(字号)}| 字号 $n$ 可为 -6--8 间的整数 ($-6 \leq n \leq 8, n \in \mathbb{Z}$), 0 和正整数分别表示初(零)号,一号,……,八号。而 -0 和负整数 $-n$ 则表示小 $n$ 号 ($0 \leq n \leq 6$).
\item 中文数字转换(调用 zhnumber 宏包实现) %\\
% \verb|\zhnumber{(number)}| 以中文格式输出数字。这里的 number 可以是整数、小数和分数。\verb|\zhdigits {⟨number⟩}| 将阿拉伯数字转换为中文数字串。
\begin{sidelst}
\zhnumber{201.5} \\
\zhnumber{1/2} \\
\zhdigits{012345} \\
\zhdigits*{012345}
\end{sidelst}
另有 \verb|\chinese{(counter)}|, 用法与 \verb|\roman| 等命令用法类似,作用在一个 \LaTeX 计数器 (counter) 上。
\end{itemize}
\end{frame}


\subsection{章节标题选讲}\label{sec:section-title}
\frame{
章节标题格式设置等属于“文档元素”主题,后续将由 ZSN 同学报告。

这里仅举两例而不作具体说明。
}

\begin{frame}
\frametitle{章节标题示例一}
\texinputlst[label=lst:chapter4]{Demos/demo-chapter-4.tex}
\end{frame}

\begin{frame}
\frametitle{章节标题示例效果图}
\begin{figure}
\centering
% trim=left bottom right top
\frame{\includegraphics[trim={9.5cm 19.8cm 9.5cm 6.6cm},clip]{demo-chapter-4}}
\caption{Listing \ref{lst:chapter4} 效果}
\label{img:4}
\end{figure}
\begin{figure}
\centering
\frame{\includegraphics[trim={4.3cm 20.5cm 5.5cm 7cm},clip]{demo-chapter-34}}
\caption{Listing \ref{lst:chapter34} 效果}
\label{img:34}
\end{figure}
\end{frame}

\begin{frame}
\frametitle{章节标题示例二}
\texinputlst[label=lst:chapter34]{Demos/demo-chapter-34.tex}
\end{frame}

\begin{frame}
\frametitle{练习}
\begin{Ex}{编译中文 \LaTeX}{compile}
请修改 Listing \ref{lst:hello}, 给出一种能够输出中文的解决方案。
\end{Ex}
\end{frame}

% \appendix
\section{补充}
\subsection[过时的中文方案]{过时的中文 \LaTeX 解决方案}
\begin{frame}
\frametitle{CTeX 套装}
\href{http://www.ctex.org/CTeX}{CTeX 套装}基于 MiKTeX 发行版,“支持 CJK, xeCJK, CCT, TY 等多种中文 \TeX 处理方式。”

\href{http://www.ctex.org/CTeXReleaseNotes}{CTeX Release Notes} 中最新版本为 v2.9.2.164 --- 2012.03.22.

\alert{一款停止维护更新的软件注定不能适应新的时代。}
\end{frame}

\begin{frame}
\frametitle{其他方案}
\begin{description}
\item[CJK 宏包] CTAN 上最后一个 \href{https://www.ctan.org/pkg/cjk}{CJK 宏包}发布版本为 4.8.4 2015-04-18. 而 \href{https://www.ctan.org/pkg/xecjk}{xeCJK 宏包}最后更新于 2018-05-01, 且当前仍有两位活跃的维护者 (Qing Lee, Leo Liu)。
\item[CCT 模板] 博文《\href{https://liam.page/2013/10/15/LaTeX-CCT-template/}{【LaTeX Tips】国内期刊 CCT 模板编译经验}》对过时的 \href{ftp://ftp.cc.ac.cn/pub/cct/}{CCT} 模板的使用有所讨论。
\item[TY] 可能过于老旧,我搜索不到相关信息。
\end{description}
\end{frame}

\subsection{关于字符编码}
\begin{frame}
\frametitle{字符编码 (encoding)}
包含中文的常见编码方式有 UTF-8 和 GBK.

GBK 编码常见于 Windows 中文版操作系统。当前版本的 Win 10 系统已经支持修改默认编码 (encoding) 为 UTF-8.

使用 \hologo{XeLaTeX}、\hologo{LuaLaTeX}、up\LaTeX 引擎编译时, \CTeX 宏集只支持 UTF-8 编码(见文档 \cite{ctex} \S 4.1 编译方式一节)。
\end{frame}

\begin{frame}[fragile]
\frametitle{注记}
使用 \hologo{XeLaTeX}/\hologo{LuaLaTeX} 引擎时,所有 Unicode 字符均可直接输入。

例如,所有中文标点,包括单/双引号 ‘\, ’ “\, ” 均直接输入。但建议英文单双引号以 \verb|`'| 或 \verb|``''| 的形式输入。

\begin{Ex}{编码转换}{encoding}
请尝试将一 GBK 编码的文件以 UTF-8 编码打开并保存。然后将新文件再次以 GBK 编码打开并保存。
\end{Ex}
\begin{alertblock}{强烈建议}
请将源代码文件以 UTF-8 编码保存。
\end{alertblock}
\end{frame}


\subsection{latexmk 工具}\label{ap:latexmk}
\begin{frame}[fragile]
\frametitle{latexmk 工具}
\begin{quote}
Latexmk completely automates the process of compiling a LaTeX document.
\end{quote}
以编译当前的文档 Chinese-beamer.tex 为例。可使用的编译方式为
\begin{bashlst}
latexmk -xelatex Chinese-beamer
\end{bashlst}
实际中依次运行了 `xelatex', `biber', `xelatex' ($\times 3$),即
\begin{bashlst}[numbers=left]
xelatex Chinese-beamer
biber Chinese-beamer
xelatex Chinese-beamer
xelatex Chinese-beamer
xelatex Chinese-beamer
\end{bashlst}
\end{frame}

\begin{frame}[fragile]
\frametitle{latexmkrc}
配合配置文件 latexmkrc, latexmk 的功能更加强大。

例如,编译当前报告(配合文档),如果只利用 latexmk 工具,至少需要
\begin{bashlst}[numbers=left]
cd Demos
xelatex demo-chapter-4
xelatex demo-chapter-34
cd ../
latexmk -xelatex Chinese
latexmk -xelatex Chinese-beamer
\end{bashlst}
如果同时利用项目中的 `.latexmkrc' 文件,则只需
\begin{bashlst}
latexmk
\end{bashlst}
\end{frame}

\subsubsection{latexmk 命令参数}
\begin{frame}
\frametitle[命令参数]{latexmk 命令参数}
\textbf{latexmk} 有很多可选参数(空格隔开,无顺序要求)。这里仅举几例:
\begin{description}
\item[-f] 即无视报错,强制 (force) 执行。
\item[-pv] 预览 (preview) 输出文件。
\item[-c] 删除 (clean up) 生成的中间文件,除了少数例外。
\item[-C] 删除包括 PDF 在内的生成文件。(有例外。)
\end{description}
\end{frame}

\begin{frame}
\frametitle{练习}
\begin{Ex}{利用 latexmk 自动编译}{latexmk}
请分别手动和使用 \textbf{latexmk} 编译 Chinese-beamer.tex 和 Chinese.tex. \\ \myhrule

提示:生成的 PDF 文档中有两个半角符号 `?'. 可以搜索自己手动编译生成的 PDF 文档中问号的数量是否多于两个初步判断编译步骤是否全部完成。此外,还可通过编译日志 Chines-beamer.log 文件及编译器的输出查询编译步骤是否欠缺。
\end{Ex}
\end{frame}


\subsection{多文件文档}
\subsubsection{标准方式}
\begin{frame}[fragile]
\frametitle{文件引用标准方式}
\verb|\input{filename}| 或 \verb|\include{filename}| 来引用纯文本文件

根据总结 \cite{input-include-tex}:
\begin{itemize}
\item \verb|\input{|\emph{filename}\verb|}| 相当于直接把文件 \emph{filename} 中的内容复制粘贴到插入该引用命令的位置。
\item \verb|\include{filename}| 可认为是在添加 \verb|\clearpage| 后再 `input'. \\
\verb|\clearpage| 强制换页。\verb|\include{}| 可以配合导言区中的 \verb|\includeonly{}| 使用。\verb|\include{}| 不能嵌套使用。
\end{itemize}
利用注释,\verb|\input{}| 也可控制只编译所需内容。
\end{frame}

\begin{frame}[fragile]
\frametitle{文件引用扩展方式}
文档 \cite{input-include-overleaf} 中提到,`import' 宏包提供的 \verb|\import{directory}{file}| 命令允许嵌套使用。

嵌套使用时,可能用到 \verb|\subimport{directory}{file}|. 此时文件夹路径(directory)是相对于被引用的文件所在目录的。而 \verb|\import| 则相对于主文件所在目录。

\begin{Ex}{多文件文档}{multi-files}
请自己完成一个多文件文档,并编译得到 PDF.
\end{Ex}
\end{frame}

\subsection*{结语}
\begin{frame}
\frametitle{鸣谢}
感谢 FZX、LFZ、SCJ 三位同学帮我审阅初稿并指出错误和不足,包括内容、命令的错误,排版的不美观,和内容上的缺漏等。

其中,根据 FZX 同学的反馈,我增加了 \S \ref{sec:practical-options} 实用命令选讲 和 \S \ref{sec:section-title} 章节标题选讲 两部分内容。根据 SCJ 同学的现场反馈,我在 article 文档中列出了 Listing \ref{lst:latexmkrc} 并增加了第 10, 11 两行(但这两行由于被注释了,不起实际作用)。
\end{frame}
  
% \subsection*{参考资料}
\begin{frame}[allowframebreaks]{参考资料}
\printbibliography[heading=none]
\end{frame}

\end{document}
\LaTeX 源代码可以存放在多个文件中。这对于较长篇幅的代码的管理是有益的。这在对长文件 debug 时作用尤为突出。

实际上,这篇文章就将所有正文内容分散在了 `Sections' 文件夹中,并在主文件 `Chinese.tex' 中以 \verb|\input{Sections/filename.tex}| 调用。
% \lstinputlisting[%
%   language=bash,
%   numbers=none,
%   numberstyle=\scriptsize,
%   frame=TB,
%   rulesep=.7pt,
%   rulesepcolor=\color{\framecolor},
%   backgroundcolor=\color{bgcolor},
%   breaklines,
%   breakautoindent=true,
%   basicstyle=\small\ttfamily,
%   commentstyle=\small,]%
%   {Sections/ls.txt}
\bashinputlst[caption=文档结构,label=lst:documents]{Sections/ls.txt}

标准用法中,常通过 \verb|\input{filename}| 或 \verb|\include{filename}| 来引用其他文件(包括但不限于 .tex 文件)。

根据总结 \cite{input-include-tex}:
\begin{itemize}
\item \verb|\input{filename}| 相当于直接把文件 filename 中的内容复制粘贴到插入该引用命令的位置。
\item \verb|\include{filename}| 可认为是在添加 \verb|\clearpage| 后再 `input'. \verb|\clearpage| 强制换页。\verb|\include{}| 可以配合导言区 (preamble) 中的 \verb|\includeonly{}| 使用(这意味着你可以只编译所需的那部分内容,从而有效降低编译时间)。\verb|\include{}| 不能嵌套使用。
\end{itemize}
另外,如果使用注释功能(行首加 \verb|%|),\verb|\input{}| 也可以方便地控制只编译所需的那部分内容。

\verb|\input{}| \verb|\include{}| 均能在导言区使用。但是,根据 \cite{input-include-overleaf}, 对于写在分开的文件中的导言区的用户自定义的命令、定义等,“正确” (The right way) 的方法是创建一个自定义的包 (package), 并保存成 `\emph{pkgname}.sty' 文件。该文件的首行为
\begin{texlst}
% pkgname.sty
\ProvidesPackage{pkgname}
\end{texlst}
并在主文件中引用:
\begin{texlst}[numbers=none]
% main.tex
\usepackage{pkgname}
\end{texlst}
注意自命名的 `pkgname' 保持一致。

文档 \cite{input-include-overleaf} 中提到,`import' 宏包提供的 \verb|\import{path/to/directory}{file}| 命令允许嵌套使用。

嵌套使用时,可能用到 \verb|\subimport{path/to/directory}{file}|. 此时文件夹路径(第一个 \verb|{}| 中的路径)是相对于被引用的文件所在目录的。而 \verb|\import| 则相对于主文件所在目录。
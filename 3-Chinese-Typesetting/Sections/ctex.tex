\CTeX 宏集手册\cite{ctex} 中写道:
\begin{quote}
\CTeX 宏集是由 \href{http://bbs.ctex.org/}{\CTeX 社区}发起并维护的 \LaTeX 宏包和文档类的集合。社区另有发布名为 \CTeX 套装的 \TeX 发行版,与本文档所述的 \CTeX 宏集并非是同一事物。
\end{quote}

该文档中 \S 4.1 编译方式一节写道:\begin{quote}
  \CTeX 宏集会根据用户使用的编译方式,在底层选择不同的中文支持方式
\end{quote}
(见表 \ref{tb:ctex})。现在我们可以说,目前使用 \LaTeX 排版中文,推荐的方案是调用 xeCJK/ctex 宏包,并相应使用 \hologo{XeLaTeX} 引擎编译生成 PDF 文件。当使用 latexmk 工具(可参见附录 \ref{ap:latexmk})自动编译时,相应添加参数 `-xelatex'.

\begin{table}
\caption{\CTeX 宏集的中文支持方式}
\label{tb:ctex}
\centering
\begin{tabular}{*{5}{c}}
\toprule
编译方式 & (pdf)\LaTeX & \hologo{XeLaTeX} & \hologo{LuaLaTeX} & up\LaTeX \\ \midrule
支持宏包 & CJK & xeCJK & LuaTeX-ja & 原生 \\ \bottomrule
\end{tabular}
\end{table}

\subsection{ctex 文档类和宏包}
ctex 文档类包括 ctexart, ctexrep, ctexbook 和 ctexbeamer 分别是对标准文档类 article, report, book 和 beamer 的封装。ctex 文档类调用了 ctex 宏包。

推荐使用 ctex 文档类的情景:文档主体为中文。

一个最简单的 ctexart 文档类实例可以是如下这样的:
\texinputlst[label=lst:demo-ctex]{Demos/demo-ctex.tex}

该宏集中现提供宏包 ctex.sty, ctexsize.sty, ctexheading.sty.

\begin{Ex}{使用 ctex 宏包}{ctex-package}
请将 Listing \ref{lst:hello} 中的文档类修改为 ctexart, 利用 xelatex 或 latexmk -xelatex 编译得到 PDF 文档。
\end{Ex}

\subsection{\textbackslash ctexset\{\}}
下面简单介绍导言区设置 \verb|\ctexset{(键值列表)}| 中的一部分键值 (见 \cite{ctex} 中 \S 5.3, 6.1).
\begin{itemize}
\item scheme = (\textbf{chinese}|\emph{plain})
\begin{description}
  \item[chinese] 调整默认字号;汉化标题名字(如“表”等);在 heading = true 时,将章节标题修改为中文样式。
  \item[plain] 仅提供中文支持功能,而不对文章版式进行任何修改。 
  \end{description}
\item linespread = (数值)\\
设置行距倍数(接受浮点值)。初始值依赖于 scheme.
\item today = (\textbf{small}|\emph{big}|\emph{old})
\begin{description}
  \item[small] 效果为“2019 年 10 月 27 日”
  \item[big] 效果为“二〇一九年十月二十七日”
  \item[old] 效果为原本的(英文) 日期格式 
\end{description}
\end{itemize}

\subsection{实用命令选讲}\label{sec:practical-options}
参见 \cite{ctex} 第 8 节 实用命令。
\begin{itemize}
\item \verb|\zihao{(字号)}| \\
字号 $n$ 可为 -6--8 间的整数 ($-6 \leq n \leq 8, n \in \mathbb{Z}$), 0 和正整数分别表示初(零)号,一号,……,八号。而 -0 和负整数 $-n$ 则表示小 $n$ 号 ($0 \leq n \leq 6$).
\item 中文数字转换 \\
通过调用 \href{http://mirrors.ctan.org/macros/latex/contrib/zhnumber/zhnumber.pdf}{zhnumber 宏包}实现。\verb|\zhnumber{(number)}| 以中文格式输出数字。这里的 number 可以是整数、小数和分数。\verb|\zhdigits {⟨number⟩}| 将阿拉伯数字转换为中文数字串。
\begin{sidelst}
\zhnumber{201.5} \\
\zhnumber{1/2} \\
\zhdigits{012345} \\
\zhdigits*{012345}
\end{sidelst}
\verb|\chinese{(counter)}| 用法与 \verb|\roman| 等命令用法类似,作用在一个 \LaTeX 计数器上。
\end{itemize}

\subsection{章节标题选讲}\label{sec:section-title}
\CTeX 宏集手册 \cite{ctex} 第 7 节 “章节标题格式设置” 写道:
\begin{quote}
章节标题名包括 part, chapter, section, subsection, subsubsection,
paragraph, subparagraph;而可用的格式包括 numbering, name, number, format, nameformat, numberformat, aftername, titleformat, aftertitle, runin, afterindent, beforeskip, afterskip, fixskip, lotskip, lofskip, indent, hang, pagestyle, break, tocline 等。
\end{quote}
以上所有选项使用 \verb|\ctexset| 命令设置。

由于章节标题格式设置等属于“文档元素”主题,后续将由 ZSN 同学报告。这里仅举两例而不作具体说明。
\texinputlst[label=lst:chapter4]{Demos/demo-chapter-4.tex}
\texinputlst[label=lst:chapter34]{Demos/demo-chapter-34.tex}

\begin{figure}
\centering
% trim=left bottom right top
\frame{\includegraphics[trim={9.5cm 19.8cm 9.5cm 6.6cm},clip]{demo-chapter-4}}
\caption{Listing \ref{lst:chapter4} 效果}
\label{img:4}
\end{figure}
\begin{figure}
\centering
\frame{\includegraphics[trim={4.3cm 20.5cm 5.5cm 7cm},clip]{demo-chapter-34}}
\caption{Listing \ref{lst:chapter34} 效果}
\label{img:34}
\end{figure}


\begin{Ex}{查阅 ctex 宏集手册}{ctex}
请查阅\href{http://mirrors.ctan.org/language/chinese/ctex/ctex.pdf}{ctex 宏集手册}中任一感兴趣内容。
\end{Ex}
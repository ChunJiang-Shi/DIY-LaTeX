待输出的 PDF 文档中包含中文时,我们推荐调用 xeCJK 宏包,并相应使用 \hologo{XeLaTeX} 编译\footnote{后文有进一步的补充方案}。

% 文件 \href{run:demo-xeCJK.tex}{demo-xeCJK.tex} 展示了 xeCJK 宏包的一种可能的使用场景。
\texinputlst[label=lst:demo-xecjk]{Demos/demo-xeCJK.tex}

\subsection{xeCJK 宏包选项选讲}
xeCJK 宏包手册 \cite{xecjk} 中写道:“xeCJK 以 ⟨key⟩=⟨var⟩ 的形式提供宏包选项,你可以在调用宏包的时候直接设置这些选项,也可以在调用宏包之后使用 \verb|\xeCJKsetup| 来设置这些选项。”

以下列举几例(见 \cite{xecjk} \S 3.1):
\begin{itemize}
\item xeCJKactive = (\textit{true}|\textbf{false}) \\
缺省状态下, xeCJK 会忽略 CJK 文字之间的空格,使用这一选项来保留它们之间的空格。
\item CJKmath = (\textit{true}|\textbf{false}) \\
是否支持在数学环境中直接输入 CJK 字符。使用这个选项后,可以直接在数学环境中输出
CJK 字符。url 宏包将一个 URL 放在一个特殊的数学环境中排版,所以如果在 \verb|\path| 等命令的路径参数中含有汉字,则需要启用这个选项,路径中的汉字才能显示。
\item AutoFallBack = (\textit{true}|\textbf{false}) \\
当文档中有个别生僻字时,可以使用这个选项,自动使用预先设置好的后备字体来输出这些生僻字。后备字体的设置方法在 3.2 节中有介绍。
\end{itemize}

Listing \ref{lst:demo-xecjk} 中使用了 `\verb|\setCJKmainfont{|\textit{font name}\verb|}|'. 这里 `SimSun' `FandolSong' 均为字体名称。\href{http://mirrors.ctan.org/macros/xetex/latex/xecjk/xeCJK.pdf#6}{xeCJK 文档 \S 3.2.1} 一节指出了通过 `fc-list' 命令查看当前系统下的已安装字体名称的方法。

推荐手动调用 xeCJK 宏包的情景:文档主体不为中文,仅局部插入中文。

\begin{Ex}{设置全局中文字体}{setCJKmainfont}
请从自己系统已安装的中文字体中,寻找一款自己喜欢的字体,利用 Listing \ref{lst:demo-xecjk} 中方式,设置成一个文档的全局中文字体。
\end{Ex}
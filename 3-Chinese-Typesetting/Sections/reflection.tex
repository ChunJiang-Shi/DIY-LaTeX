\subsection{数学相关}
前两次报告中,FZX、LH、GR 三位同学详细地介绍了 \LaTeX 概述、数学环境、AMS 宏集、字体等。请回忆相关内容。

我在此仅就由 amsmath 宏包提供的三个功能作一点补充:
\begin{itemize}
\item 带框数学环境 \verb|\boxed{}| \verb|\framebox[]{}|. 以 \verb|\boxed{}| 为例:
\begin{vertlst}
% \usepackage{amsmath}
% \newcommand{\iu}{\mathrm{i}\mkern1mu}
% \newcommand{\e}{\mathrm{e}}

\boxed{\e^{\iu \theta} = \cos{\theta} + \iu \sin{\theta}}
\end{vertlst}
\item \textsf{gathered}, \textsf{aligned}, \textsf{alignedat} 嵌套数学环境。以 gathered 为例:
\begin{vertlst}
% \usepackage{amsmath}

\begin{equation}
\begin{gathered}[]
  [f, f] = 0, \\
  [f, g] + [g, f] = 0, \\
  [f, [g, h]] + [g, [h, f]] + [h, [f, g]] = 0.
\end{gathered}
\end{equation} 
\end{vertlst}
\item cases 环境。效果如下:
\begin{sidelst}
\begin{equation}
  \delta_{ij} =
  \begin{cases}
    0, \, & i \ne j \\
    1, & 1 = j
  \end{cases}
\end{equation}
\end{sidelst}
\end{itemize}


\subsection{提问相关}
第三次报告中,LFZ 同学就提问相关内容作了生动有趣的介绍。总地来说,遇到问题可先查阅宏包文档或网页已有资源。准备提问前应首先整理好最小工作示例 (Minimal Working Example, \emph{MWE}),以简洁且他人能够复现问题为标准。提问的途径很多,比如 \TeX 相关的问答网站、论坛,以及多人群组,还可以向身边的人提问。